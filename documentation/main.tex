\documentclass[12pt,a4paper]{article}
\usepackage[utf8]{inputenc}
\usepackage{geometry}
\geometry{margin=1in}

\title{Yutex Tech Stack}
\author{Yousuf Uyghur}
\date{}

\begin{document}

\maketitle

\section*{Yutex Tech Stack}

\section*{Overview}
App is a one-stop solution for latex editors. Don't need to manually install the miktex, strawberry perl and other engines. Also it works locally and does not produce tons of 
extra files, a all-in-one application for latex. 

\section*{Frontend}
\begin{itemize}
    \item \textbf{SvelteKit}  
    Modern UI framework for reactive, fast frontend.
    \item \textbf{Adapter-Static}  
    Compiles SvelteKit into static files (\texttt{/build}) so Tauri can ship them.
    \item \textbf{Vite (bundler)}  
    Powers dev server and build pipeline.
\end{itemize}

\section*{Desktop Shell}
\begin{itemize}
    \item \textbf{Tauri}  
    Wrapper that turns your web frontend into a native desktop app.  
    Bridges frontend $\leftrightarrow$ Rust backend.
\end{itemize}

\section*{Backend}
\begin{itemize}
    \item \textbf{Rust (via Tauri commands)}  
    Handles system-level logic such as file access and LaTeX compilation.  
    Rust is secure, fast, and integrates tightly with Tauri.
\end{itemize}

\section*{LaTeX Engine}
\begin{itemize}
    \item \textbf{Tectonic (Rust-based)}  
    Modern LaTeX engine that downloads required packages automatically.  
    Replaces MiKTeX/TeX Live (avoids bloated dependencies).  
    Invoked by the Rust backend to compile \texttt{.tex} into \texttt{.pdf}.
\end{itemize}

\section*{Editor}
\begin{itemize}
    \item \textbf{Monaco Editor}
    Provides rich text/code editing with LaTeX syntax highlighting.  
    Similar to what Overleaf uses.
\end{itemize}

\section*{Summary Flow}
\begin{enumerate}
    \item User writes LaTeX in Monaco.
    \item Click \texttt{Compile} → Frontend calls a Rust Tauri command.
    \item Rust command runs Tectonic → produces PDF.
    \item PDF is shown back in the app, PDF is displayed live and automatic compilation.
\end{enumerate}

\end{document}
